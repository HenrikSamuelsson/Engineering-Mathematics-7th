\documentclass[fleqn]{article}
\usepackage{enumitem}
\usepackage{amsmath}

\begin{document}

\begin{enumerate}[label=\textbf{\arabic*.}]

%frame 2
\setcounter{enumi}{1}
\item The numbers $-10$, $4$, $0$, $-13$ are of a type called integers.

%frame 3
\item

\begin{enumerate}[label=\textbf{(\alph*)}]
\item $-3 > -6$
\item $2 > -4$
\item $-7 < 12$
\end{enumerate} 

%frame 5
\setcounter{enumi}{4}
\item

\begin{enumerate}[label=\textbf{(\alph*)}]
\item $8 + (-3) = 8 - 3 = 5$
\item $9 - (-6) = 9 + 6 = 15$
\item $(-14) - (-7) = -14 + 7 = -7$
\end{enumerate}

%frame 7
\setcounter{enumi}{6}
\item

\begin{enumerate}[label=\textbf{(\alph*)}]
\item $(-5) \times 3 = -15$
\item $12 \div (-6) = -2$
\item $(-2) \times (-8) = 16$
\item $(-14) \div (-7) = 2$
\end{enumerate}

%frame 9
\setcounter{enumi}{8}
\item $34 + 10 \div (2 - 3) \times 5 = 
34 + 10 \div (-1) \times 5 = 
34 -10 \times 5 = 
34 - 50 = -16$

%frame 14
\setcounter{enumi}{13}
\item Some numbers and the rounding of these numbers to the nearest 10, 100 and 1000. 
\begin{enumerate}[label=\textbf{(\alph*)}]
\item $ 1846, 1850, 1800, 2000 $
\item $ -638, -640, -600, -1000 $
\item $ 445, 450, 400, 0 $
\end{enumerate}

\end{enumerate}

\end{document}