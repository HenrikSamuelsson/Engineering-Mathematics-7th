\documentclass[fleqn]{article}
\usepackage{enumitem}
\usepackage{amsmath}
\usepackage{fancyhdr}
 
\pagestyle{fancy}
\fancyhf{}
\lhead{Henrik Samuelsson}
\rhead{henrik.samuelsson@gmail.com}

\begin{document}
\section*{Solutions Programme F.1 Arithmetic}
Solutions to exercises from the book Engineering Mathematics 7th edition. The book is divided into frames and the numbers of the exercises refers to these frames.

\begin{enumerate}[label=\textbf{\arabic*.}]

%frame 2
\setcounter{enumi}{1}
\item The numbers $-10$, $4$, $0$, $-13$ are of a type called integers.

%frame 3
\item

\begin{enumerate}[label=\textbf{(\alph*)}]
\item $-3 > -6$
\item $2 > -4$
\item $-7 < 12$
\end{enumerate} 

%frame 5
\setcounter{enumi}{4}
\item

\begin{enumerate}[label=\textbf{(\alph*)}]
\item $8 + (-3) = 8 - 3 = 5$
\item $9 - (-6) = 9 + 6 = 15$
\item $(-14) - (-7) = -14 + 7 = -7$
\end{enumerate}

%frame 7
\setcounter{enumi}{6}
\item

\begin{enumerate}[label=\textbf{(\alph*)}]
\item $(-5) \times 3 = -15$
\item $12 \div (-6) = -2$
\item $(-2) \times (-8) = 16$
\item $(-14) \div (-7) = 2$
\end{enumerate}

%frame 9
\setcounter{enumi}{8}
\item $34 + 10 \div (2 - 3) \times 5 = 
34 + 10 \div (-1) \times 5 = 
34 -10 \times 5 = 
34 - 50 = -16$

%frame 13
\setcounter{enumi}{12}
\item Some numbers and the rounding of these numbers to the nearest 10, 100 and 1000. 
\begin{enumerate}[label=\textbf{(\alph*)}]
\item $ 1846, 1850, 1800, 2000 $
\item $ -638, -640, -600, -1000 $
\item $ 445, 450, 400, 0 $
\end{enumerate}

%frame 14
\setcounter{enumi}{13}
\item 
\begin{enumerate}[label=\textbf{(\alph*)}]
\item $ 18 \times 21 - 19 \div 11 \approx 20 \times 20 - 20 \div 10 = 398 $
\item $ 99 \div 101 - 49 \times 8 \approx 100 \div 100 - 50 \times 10 = -499 $
\end{enumerate}

%frame 17
\setcounter{enumi}{16}
\item This frame holds multiple review exercises that follow below.

\begin{enumerate}[label=\textbf{\arabic*.}]
\item

\begin{enumerate}[label=\textbf{(\alph*)}]
\item $ -1 > -6 $
\item $ 5 > -29 $
\item $ -14 < 7 $ 
\end{enumerate}

\item

\begin{enumerate}[label = \textbf{(\alph*)}]
\item $ 16 - 12 \times 4 + 8 \div 2 = 16 - 48 + 4 = -28 $
\item $ (16 - 12) \times (4 + 8) \div 2 = 4 \times 12 \div 2 = 24 $
\item $ 9 - 3(17 +5[5 - 7]) = 9 - 3(17 - 10) = 9 - 21 = -12$
\item $ 8(3[2 + 4] - 2[5 + 7]) = 8(3 \times 6 - 2 \times 12) = 8(18 - 24) = 8(-6) = -48$ 
\end{enumerate}

\item
\begin{enumerate}[label = \textbf{(\alph*)}]

\item
Show that: \\
$ 6 - (3 - 2) \neq (6 -3) - 2 $

Proof: \\
$ LHS = 6 - (3 - 2) = 6 - 1 = 5 $ \\
$ RHS = (6 - 3) - 2 = 3 - 2 = 1 $ \\
$ LHS \neq RHS $ 

\item
Show that: \\
$ 100 \div (10 \div 5) \neq (100 \div 10) \div 5 $

Proof: \\
$ LHS = 100 \div (10 \div 5)) = 100 \div 2 = 50 $ \\
$ RHS = (100 \div 10) \div 5 = 10 \div 5 = 2$ \\
$ LHS \neq RHS $

\item
Show that: \\
$ 24 \div (2 - 6) \neq 24 \div 2 - 24 \div 6 $

Proof: \\
$ LHS = 24 \div (2 - 6) = 24 \div (-4) = -6 $ \\
$ RHS = 24 \div 2 - 24 \div 6 = 12 - 4 = 8$ \\
$ LHS \neq RHS $

\end{enumerate}
\item Some numbers and the rounding of these numbers to the nearest 10, 100 and 1000. 
\begin{enumerate}[label=\textbf{(\alph*)}]
\item $ 2562, 2560, 2600, 3000 $
\item $ 1500, 1500, 1500, 2000 $
\item $ -3451, -3450, -3500, -3000 $
\item $ -14525, -14530, -14500, -15000 $
\end{enumerate}

\end{enumerate}
%frame 17 end

%frame 19
\setcounter{enumi}{18}
\item
\begin{enumerate}[label=\textbf{(\alph*)}]
\item $ 12 = 1 \times 12 = 2 \times 6 = 3 \times 4 $ \\
So the factors of 12 are 1, 2, 3, 4, 6, 12.
\item $ 25 = 1 \times 25 = 5 \times 5 $ \\
So the factors of 25 are 1, 5, 25.
\item $ 17 = 1 \times 17 $ \\
So the factors of 17 are 1, 17. 
\end{enumerate}

\end{enumerate}


\end{document}
