\documentclass[fleqn]{article}
\usepackage{enumitem}
\setlist[enumerate]{itemsep=1.2em} %increases vertical space between items
\usepackage{amsmath}
\usepackage{fancyhdr}
 
\pagestyle{fancy}
\fancyhf{}
\lhead{Henrik Samuelsson}
\rhead{henrik.samuelsson@gmail.com}

\begin{document}
\section*{Solutions Programme F.1 Arithmetic}
Solutions to exercises from the book Engineering Mathematics 7th edition. The book is divided into frames and the numbering of the following exercises refers to the numbers of these frames.

\begin{enumerate}[label=\textbf{\arabic*.},labelsep=2em]

%frame 2
\setcounter{enumi}{1}
\item
The numbers $-10$, $4$, $0$, $-13$ are of a type called integers.

%frame 3
\item

\begin{enumerate}[label=\textbf{(\alph*)},labelsep=2em]
\item $-3 > -6$
\item $2 > -4$
\item $-7 < 12$
\end{enumerate} 

%frame 5
\setcounter{enumi}{4}
\item

\begin{enumerate}[label=\textbf{(\alph*)}]
\item $8 + (-3) = 8 - 3 = 5$
\item $9 - (-6) = 9 + 6 = 15$
\item $(-14) - (-7) = -14 + 7 = -7$
\end{enumerate}

%frame 7
\setcounter{enumi}{6}
\item
\begin{enumerate}[label=\textbf{(\alph*)}]
\item $(-5) \times 3 = -15$
\item $12 \div (-6) = -2$
\item $(-2) \times (-8) = 16$
\item $(-14) \div (-7) = 2$
\end{enumerate}

%frame 9
\setcounter{enumi}{8} 
\item
$34 + 10 \div (2 - 3) \times 5 = 
34 + 10 \div (-1) \times 5 = 
34 -10 \times 5 = 
34 - 50 = -16$

%frame 13
\setcounter{enumi}{12}

\item Some numbers and the rounding of these numbers to the nearest 10, 100 and 1000. 

\begin{enumerate}[label=\textbf{(\alph*)}]
\item $ 1846, 1850, 1800, 2000 $
\item $ -638, -640, -600, -1000 $
\item $ 445, 450, 400, 0 $
\end{enumerate}

%frame 14
\setcounter{enumi}{13}
\item 
\begin{enumerate}[label=\textbf{(\alph*)}]
\item $ 18 \times 21 - 19 \div 11 \approx 20 \times 20 - 20 \div 10 = 398 $
\item $ 99 \div 101 - 49 \times 8 \approx 100 \div 100 - 50 \times 10 = -499 $
\end{enumerate}

%frame 17
\setcounter{enumi}{16}
\item This frame holds multiple review exercises that follow below.

\begin{enumerate}[label=\textbf{\arabic*.} ,labelsep=2em]
\item

\begin{enumerate}[label=\textbf{(\alph*)},labelsep=2em]
\item $ -1 > -6 $
\item $ 5 > -29 $
\item $ -14 < 7 $ 
\end{enumerate}

\item

\begin{enumerate}[label = \textbf{(\alph*)},labelsep=2em]
\item $ 16 - 12 \times 4 + 8 \div 2 = 16 - 48 + 4 = -28 $
\item $ (16 - 12) \times (4 + 8) \div 2 = 4 \times 12 \div 2 = 24 $
\item $ 9 - 3(17 +5[5 - 7]) = 9 - 3(17 - 10) = 9 - 21 = -12$
\item $ 8(3[2 + 4] - 2[5 + 7]) = 8(3 \times 6 - 2 \times 12) = 8(18 - 24) = 8(-6) = -48$ 
\end{enumerate}

\item
\begin{enumerate}[label = \textbf{(\alph*)},labelsep=2em]

\item
Show that: \\
$ 6 - (3 - 2) \neq (6 -3) - 2 $

Proof: \\
$ LHS = 6 - (3 - 2) = 6 - 1 = 5 $ \\
$ RHS = (6 - 3) - 2 = 3 - 2 = 1 $ \\
$ LHS \neq RHS $ 

\item
Show that: \\
$ 100 \div (10 \div 5) \neq (100 \div 10) \div 5 $

Proof: \\
$ LHS = 100 \div (10 \div 5)) = 100 \div 2 = 50 $ \\
$ RHS = (100 \div 10) \div 5 = 10 \div 5 = 2$ \\
$ LHS \neq RHS $

\item
Show that: \\
$ 24 \div (2 - 6) \neq 24 \div 2 - 24 \div 6 $

Proof: \\
$ LHS = 24 \div (2 - 6) = 24 \div (-4) = -6 $ \\
$ RHS = 24 \div 2 - 24 \div 6 = 12 - 4 = 8$ \\
$ LHS \neq RHS $

\end{enumerate}
\item Some numbers and the rounding of these numbers to the nearest 10, 100 and 1000. 
\begin{enumerate}[label=\textbf{(\alph*)},labelsep=2em]
\item $ 2562, 2560, 2600, 3000 $
\item $ 1500, 1500, 1500, 2000 $
\item $ -3451, -3450, -3500, -3000 $
\item $ -14525, -14530, -14500, -15000 $
\end{enumerate}

\end{enumerate}
%frame 17 end

%frame 19
\setcounter{enumi}{18}
\item
\begin{enumerate}[label=\textbf{(\alph*)},labelsep=2em]
\item $ 12 = 1 \times 12 = 2 \times 6 = 3 \times 4 $ \\
So the factors of 12 are 1, 2, 3, 4, 6, 12.
\item $ 25 = 1 \times 25 = 5 \times 5 $ \\
So the factors of 25 are 1, 5, 25.
\item $ 17 = 1 \times 17 $ \\
So the factors of 17 are 1, 17. 
\end{enumerate}

%frame 21
\setcounter{enumi}{20}
\item
\begin{enumerate}[label=\textbf{(\alph*)},labelsep=2em]
\item $ 84 = 2 \times 42 = 2 \times 2 \times 21 = 2 \times 2 \times 3 \times 7 $
\item $ 512 = 2^9 = 2 \times 2 \times 2 \times 2 \times 2 \times 2 \times 2 \times 2 \times 2 $ \\
\end{enumerate}

%frame 23
\setcounter{enumi}{22}
\item
The prime factorizations of 84 and 512 are

$84 = 2 \times 2 \times 3 \times 7$

$ 512 = 2 \times 2 \times 2 \times 2 \times 2 \times 2 \times 2 \times 2 \times 2 $

The highest common factor of 84 and 512 is hence

$ HCF = 2 \times 2 = 4 $

And the lowest common multiple is

$ LCF = 2 \times 2 \times 2 \times 2 \times 2 \times 2 \times 2 \times 2 \times 2 \times 3 \times 7 
= 10752 $

%frame 26
\setcounter{enumi}{25}
\item This frame holds multiple review exercises that follow below.

\begin{enumerate}[label=\textbf{\arabic*.},labelsep=2em]
\item
Repeated integer division by increasingly bigger numbers gives the following products of prime factors. 

\begin{enumerate}[label=\textbf{(\alph*)},labelsep=2em]
\item $ 429 = 3 \times 11 \times 13 $
\item $ 1820 = 2 \times 2 \times 5 \times 7 \times 13 $
\item $ 2992 = 2 \times 2 \times 2 \times 2 \times 11 \times 17 $
\item $ 3185 = 5 \times 7 \times 7 \times 13 $
\end{enumerate}

\item

\begin{enumerate}[label=\textbf{(\alph*)},labelsep=2em]
\item
The prime factorizations of 63 and 42 are

$ 63 = 3 \times 3 \times 7 $

$ 42 = 2 \times 3 \times 7 $

The highest common factor of 63 and 42 is hence

$ HCF = 3 \times 7 = 21 $

And the lowest common multiple is

$ LCF = 2 \times 3 \times 3 \times 7 = 126 $
\item 
The prime factorization of 34 and 92 are

$ 34 = 2 \times 17$

$ 92 = 2 \times 2 \times 23 $

The highest common factor of 34 and 92 is hence

$ HCF = 2 $

And the lowest common multiple is

$ LCF = 2 \times 2 \times 17 \times 23 = 1564 $

\end{enumerate}

\end{enumerate} %end of frame 26

%frame 28
\setcounter{enumi}{27}
\item
An example of a proper fraction is $ \dfrac{-8}{11} $

%frame 30
\setcounter{enumi}{29}
\item
$ \dfrac{5}{9} \times \dfrac{2}{7} = \dfrac{5 \times 2}{ 9 \times 7} = \dfrac{10}{63}$

%frame 33
\setcounter{enumi}{32}
\item
$ \dfrac{3}{8}\ of\ \dfrac{5}{7} = \dfrac{3}{8} \times \dfrac{5}{7} = \dfrac{3 \times 5}{ 8 \times 7} = \dfrac{15}{56}$

%frame 34
\setcounter{enumi}{33}
\item
$ \dfrac{7}{5} $ and $ \dfrac{28}{20} $ are equivalent fractions because $ \dfrac{7}{5} \times \dfrac{4}{4} = \dfrac{7 \times 4}{5 \times 4} = \dfrac{28}{20} $

%frame 35
\setcounter{enumi}{34}
\item
$ \dfrac{84}{108} = \dfrac{42 \times 2}{54 \times 2} = \dfrac{42}{54} = \dfrac{21 \times 2}{27 \times 2} = \dfrac{21}{27} = \dfrac{7 \times 3}{ 9 \times 3} = \dfrac{7}{9} $

%frame 37
\setcounter{enumi}{36}
\item
$ \dfrac{7}{13} \div \dfrac{3}{4} = \dfrac{7}{13} \times \dfrac{4}{3} = \dfrac{28}{39}$

%frame 38
\setcounter{enumi}{37}
\item
The reciprocal of $ \dfrac{17}{4} $ is $ \dfrac{4}{17} $

%frame 39
\setcounter{enumi}{38}
\item
The reciprocal of $ -5 $ is $ -\dfrac{1}{5} $

%frame 41
\setcounter{enumi}{40}
\item
$ \dfrac{5}{9} + \dfrac{1}{6} = \dfrac{10}{18} + \dfrac{3}{18} = \dfrac{13}{18} $

%frame 43
\setcounter{enumi}{42}
\item
$ \dfrac{11}{15} - \dfrac{2}{3} = \dfrac{11}{15} - \dfrac{10}{15} = \dfrac{1}{15} $

%frame 47
\setcounter{enumi}{46}
\item
We have $ \dfrac{3}{4} = \dfrac{9}{12} $ of A, $ \dfrac{1}{6} = \dfrac{2}{12} $ of B and $ \dfrac{1}{12} $ of C in the compound.

This means that the ratio in the compound is 

$ 9:2:1 $

%frame 49
\setcounter{enumi}{48}
\item
The percentage of cars that are red is

$ \dfrac{13}{100}=13\%$

%frame 50
\setcounter{enumi}{49}
\item
$ \dfrac{12}{25} = \left(\dfrac{12}{25}\times100\right)\% = \left(\dfrac{12}{25}\times25\times4\right)\% = \left(12\times4\right)\%=48\%$

%frame 52
\setcounter{enumi}{51}
\item
$ \dfrac{8}{100}\times25 = \dfrac{2\times4}{4\times25}\times25 = 2 $

%frame 54
\setcounter{enumi}{53}
\item This frame holds multiple review exercises that follow below.

\begin{enumerate}[label=\textbf{\arabic*.},labelsep=2em]

\item
Reduction of fractions to their lowest terms.

\begin{enumerate}[label=\textbf{(\alph*)},labelsep=2em]

\item $ \dfrac{24}{30} = \dfrac{2 \times 2 \times 2 \times 3}{2 \times 3 \times 5} = \dfrac{2 \times 2}{5} = \dfrac{4}{5} $

\item $ \dfrac{72}{15} = \dfrac{2 \times 2 \times 2 \times 3 \times 3}{3 \times 5} = \dfrac{2 \times 2 \times 2 \times 3}{5} = \dfrac{24}{5} $

\item $ -\dfrac{52}{65} = -\dfrac{2 \times 2 \times 13}{5 \times 13} = -\dfrac{2 \times 2}{5} = -\dfrac{4}{5} $

\item $ \dfrac{32}{8} = -\dfrac{4 \times 8}{8} = 4 $

\end{enumerate}

\item
Fraction evaluation.

\begin{enumerate}[label=\textbf{(\alph*)},labelsep=2em]

\item 
$ 
	\dfrac{5}{9} \times \dfrac{2}{5} = 
	\dfrac{5 \times 2}{ 9 \times 5} = 
	\dfrac{2}{9}
$

\item 
$ 
	\dfrac{13}{25} \div \dfrac{2}{15} = 
	\dfrac{13}{25} \times \dfrac{15}{2} =
	\dfrac{13}{5 \times 5} \times \dfrac{3 \times 5}{2} = 
	\dfrac{13 \times 3 \times 5}{5 \times 5 \times 2} = 
	\dfrac{39}{10}
$

\item
$
	\dfrac{5}{9} + \dfrac{3}{14} =
	\dfrac{5 \times 14}{9 \times 14} + \dfrac{3 \times 9}{14 \times 9} =
	\dfrac{70}{126} + \dfrac{27}{126} =
	\dfrac{97}{126}
$

\item
$
	\dfrac{3}{8} - \dfrac{2}{5} =
	\dfrac{3 \times 5}{8 \times 5} - \dfrac{2 \times 8}{5 \times 8}=
	\dfrac{15}{40} - \dfrac{16}{40} =
	- \dfrac{1}{40}
$

\item
$
	\dfrac{12}{7} \times \left( - \dfrac{3}{5} \right) =
	- \left( \dfrac{12}{7} \times \dfrac{3}{5} \right) =
	- \dfrac{12 \times 3}{7 \times 5} =
	- \dfrac{36}{35}
$

\item
$
	\left( -\dfrac{3}{4} \right) \div \left( -\dfrac{12}{7} \right) =
	\left( -\dfrac{3}{4} \right) \times \left( -\dfrac{7}{12} \right) =
	\dfrac {(-3) \times (-7)}{4 \times 12} =
	\dfrac {3 \times 7}{4 \times 12} = \\[8pt]
	\dfrac{3 \times 7}{4 \times 3 \times 4} = 
	\dfrac{7}{4 \times 4} =
	\dfrac{7}{16}
$

\item
$
	\dfrac{19}{2} + \dfrac{7}{4} =
	\dfrac{38}{4} + \dfrac{7}{4} =
	\dfrac{45}{4}
$

\item
$
	\dfrac{1}{4} - \dfrac{3}{8} =
	\dfrac{2}{8} - \dfrac{3}{8} =
	-\dfrac{1}{8}
$

\end{enumerate}

\end{enumerate} %end of frame 54

\end{enumerate}

\end{document}
